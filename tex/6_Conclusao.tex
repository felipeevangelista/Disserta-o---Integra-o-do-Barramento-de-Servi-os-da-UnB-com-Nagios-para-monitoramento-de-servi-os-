\label{conclusao}

O \acrshort{CPD} vem trabalhando nos últimos anos na modernização dos \textit{softwares} por ele gerenciados, desde os sistemas que eram hospedados nos \textit{mainframes}, que precisaram ser migrados para outras plataformas por causa da virada do milênio (\textit{bug} do milênio) até os sistemas desenvolvidos em tecnologias mais atuais. 

a partir da experiencia adquirida ao longo dos anos, com erros e acertos em definições arquiteturais, finalmente por meio de pesquisas e fundamentação teórica, decidiu-se pela implementação e utilização de uma arquitetura \acrshort{SOA} que é leve, flexível, coesa e com baixo acoplamento, possui seus desafios e \textit{tradeoffs}, mas que devido a sua padronização é muito bem aceita no mercado, instituições e empresas de desenvolvimento de \textit{software} e proporcionando um desenvolvimento mais ágil dos serviços, por parte dos desenvolvedores. No \acrshort{CPD} proporcionou o desenvolvimento de vários serviços e sistemas consumidores desses serviços, impactando em uma maior utilização e solicitação de implementação de novos serviços.      

Este trabalho proporcionou o estudo, análise e pesquisa de um conjunto de soluções tecnológicas que possibilitaram implementação e implantação de ferramentas que auxiliassem na realização do monitoramento dos recursos do barramento de serviços por meio da integração com uma ferramenta de monitoramento via o protocolo \acrshort{SNMP}, seguindo  abordagens citadas no capítulo \ref{fundamentacao_teorica}, que foi possível após a realização do mapeamento sistemático apresentado no capítulo \ref{mapeamento_sistematico} subsidiando a utilização da solução para a realização do monitoramento.  

 A realização deste trabalho facilitou a utilização de novas definições de métricas para a execução do monitoramento, visto que o modo para gerar as informações oriundas da instrumentação do barramento de serviços necessitavam de adequações para o recebimento ou interceptação das informações pela ferramenta de monitoramento. Essa solução foi desenvolvida para utilização inicialmente no \acrshort{CPD} da \acrshort{UnB}, mas poderá ser utilizado em outras instituições que fizerem o uso do Erlangms. 

O trabalho tem a intenção de suprir uma necessidade em relação a gestão, acompanhamento e monitoramento dos serviços disponibilizados pelo barramento de serviços, que nos dias atuais já englobam um grande conjunto de serviços (\textit{web services}) que compõem os sistemas estruturantes da \acrshort{UnB}.


%%%%%%%%%%%%%%%%%%%%%%%%%%%%%%%%%%%%%%%%%%%%%%%%%%%%%%%%%%%%%%%%%%%%%%%%%%%%%%%
\section{Contribuições}

A intenção deste trabalho é contribuir com a apoio as soluções de monitoramento executados pelo \acrshort{CPD} da \acrshort{UnB} e elenca com essencial a realização da integração das aplicações Erlangms, Exometer e Nagios\textsuperscript{\textregistered} para a realização do monitoramento por meio do protocolo \acrshort{SNMP}, com a implementação de um módulo capaz de coletar, armazenar e transmitir informações para a execução do monitoramento.

A realização do mapeamento sistemático foi mais um ponto identificado como contribuição para a orientação da seleção e utilização das ferramentas, principais soluções e desafios para a realização do monitoramento dos recursos do barramento de serviços por meio do protocolo \acrshort{SNMP}.

Por fim, a implementação do módulo \acrshort{SNMP} no barramento de serviços Erlangms, proporciona o barramento de serviços utilizar qualquer ferramenta de monitoramento que funcione com o protocolo \acrshort{SNMP}.




%%%%%%%%%%%%%%%%%%%%%%%%%%%%%%%%%%%%%%%%%%%%%%%%%%%%%%%%%%%%%%%%%%%%%%%%%%%%%%%
\section{Trabalhos Futuros}





%%%%%%%%%%%%%%%%%%%%%%%%%%%%%%%%%%%%%%%%%%%%%%%%%%%%%%%%%%%%%%%%%%%%%%%%%%%%%%%