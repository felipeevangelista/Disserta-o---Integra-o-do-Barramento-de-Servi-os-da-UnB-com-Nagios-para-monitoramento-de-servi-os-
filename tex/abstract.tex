The implementation of services and micro services for distributed system applications has been gaining ground in the choice of software architects for the application and definition of a Service Oriented Architecture (SOA), due to the possibility of organization and use of standards that allow and provide developers with modularity. , ease of maintenance, flexibility in the development of services (web services), reuse and interoperability of services developed. The Center for Informatics (CPD) of the University of Brasília (UnB) is currently working on various software automation processes, ranging from the maintenance of legacy systems, through the development of new applications to the deployment of purchased software, ie. working with various systems-related technology fronts. The proposal adopted by CPD is the implementation of the required software and the development of web services to modernize and integrate systems in order to provide web services capable of replacing some functionality of existing systems (legacy systems), as well as ensuring interoperability between them, as the systems have several functionalities. Because of this, many web services have necessarily been developed, and because of the architectural flexibility and CPD's goal of deploying, deploying, and integrating, many other web services are likely to be developed. However, as a problem, due to the large number of web services developed, the management and monitoring of these web services is a problem. The work aims to investigate solutions and tools for the implementation of Distributed Systems Monitoring using the Simple Network Management Protocol (SNMP), with the intention of monitoring the operation of web services, in order to allow monitoring and management of metrics, such as number of requests from a given web service and its availability.