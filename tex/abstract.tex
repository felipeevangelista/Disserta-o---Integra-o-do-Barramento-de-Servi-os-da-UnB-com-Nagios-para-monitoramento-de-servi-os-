Implementing services and micro services for distributed system applications using a Service Oriented Architecture (SOA) provides the ability to use development standards, facilitate maintenance, flexibly develop services, and enable interoperability of services and systems. The Center for Informatics (CPD) of the University of Brasilia (UnB) works with various software automation processes, from the maintenance of legacy systems, through the development of new applications to the deployment of purchased software, with various technological fronts related to systems. Monitoring and monitoring the operation of services, microservices and systems is essential. This work is exploratory and seeks to investigate solutions and tools for the implementation and implementation of distributed services and systems monitoring of the University of Brasilia (UnB) based on a systematic monitoring of distributed systems monitoring in order to support monitoring during their operation. services and applications. Based on the systematic mapping, a solution was obtained that made it possible to create a service bus module at the University of Brasília (UnB) to perform the monitoring. In this work, tests and simulations were performed in the solution that allowed to verify and validate the integration between the service bus and a monitoring tool through a protocol.