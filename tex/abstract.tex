The Implementation services and microservices for distributed system applications using a Service Oriented Architecture (SOA) allows to use development standards facilitate maintenance flexibly develop services and enable interoperability of services and systems. Computer Center (CPD) of the University of Brasilia (UnB) works with several softwares automation processes, from the maintenance of legacy systems, through the development of new applications to the deployment of purchased softwares, with several systems related technological fronts. Mark and monitor the functioning of services, microservices and systems is essential. This work is exploratory and seeks to investigate solutions and tools for the implementation of monitoring of distributed services and systems of the University of Brasilia (UnB), through a literature review. With a theoretical basis, a model was obtained, which was implemented as a service bus monitoring module at the University of Brasilia (UnB). In this work, simulations were performed on the solution that allowed to analyze the integration of the service bus with the monitoring tool through the proposed solution.