Deploying services and microservices for distributed system applications using a Service Oriented Architecture (SOA) provides the ability to use development standards, facilitate maintenance, flexibly develop services, and enable interoperability of services and systems. . Centrode Informática (CPD) of the University of Brasilia (UnB) works with various software automation processes, from the maintenance of legacy systems, through the development of new applications to the deployment of purchased software, with various systems-related technological fronts. Monitoring and monitoring the functioning of services, microservices and systems is essential. This work is exploratory and seeks to investigate solutions and tools for the implementation and implementation of monitoring of distributed services and systems of the University of Brasilia (UnB) based on a systematic mapping of distributed systems monitoring to support the monitoring the execution in production of services and applications. Based on the systematic mapping, a model was implemented and implemented as a service bus monitoring module at the University of Brasilia (UnB). In this work, tests and simulations were performed in the solution that allowed to analyze and validate the integration of the service bus with the monitoring tool through the proposed solution.