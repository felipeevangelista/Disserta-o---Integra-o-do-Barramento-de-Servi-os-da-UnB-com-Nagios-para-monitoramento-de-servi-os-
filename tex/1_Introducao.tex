\label{Introducao}

A modernização de sistemas legados é um tema que vem sendo cada vez mais discutido na Universidade de Brasília (UnB). Um sistema legado pode ser classificado como \textit{software} que não possui documentação, que é de difícil manutenção e também possui defasagem tecnológica, mas um ponto importante, é que normalmente é essencial para organização. Com o processo de modernização espera-se reduzir os custos com a manutenção dos sistemas legados e aumentar a integração dos fluxos de negócios entre os sistemas\cite{Agilar}.

O Centro de Informática (CPD) da UnB, órgão responsável por desenvolver, gerenciar e manter os sistemas novos e legados, atualmente vem trabalhando na implantação de uma arquitetura orientada a serviços (SOA), como modelo de modernização dos \textit{softwares} da UnB utilizando o processo  \textit{Software Modernization through Service Oriented Computing} (SMSOC) para a implementação dos serviços conectando-os  através de um barramento de serviços (ESB) que utiliza a arquitetura \textit{Representational State Transfer} (REST).

No entanto, a facilidade que a arquitetura proporciona na modernização dos sistemas legados da UnB, pode dificultar o gerenciamento e monitoramento dos serviços que serão implementados, devido ao modelo negocial de cada sistema, em alguns casos poderão haver situações em que vários serviços e micros serviços serão necessários ao negócio do sistema. Diante desse cenário o presente trabalho de pesquisa busca apresentar uma solução para o monitoramento dos serviço implementados, com a implementação do Protocolo de monitoramento \textit{Simple Network Management Protocol} (SNMP) como um serviço de monitoramento com o objetivo de facilitar o monitoramento dos serviços com a utilização do serviço implementado para integrar com ferramentas de monitoramento, que utilizam o protocolo SNMP para coletar informações, além de validar e verificar detectar condições de falhas nos serviços em execução.

%%%%%%%%%%%%%%%%%%%%%%%%%%%%%%%%%%%%%%%

\section{Justificativa}

Com o aumento da implementação e disponibilização de serviços na UnB, foi identificada a necessidade de um efetivo monitoramento dos serviços, através da coleta de dados ou informações extraídas das requisições. Para gerenciar o monitoramento são necessária ferramentas para um controle mais fácil e objetivo, atualmente o CPD utiliza ferramentas de mercado com esse propósito, essas ferramentas ou  plataformas utilizadas para acompanhamento e monitoramento da infraestrutura de redes da UnB, já a parte de sistemas e serviços não são monitoradas de forma precisa ou que possa trazer informações com certa relevância, pois não há uma comunicação ou integração dos sistemas e serviço de forma apropriada, o que implica em um déficit no acompanhamento e monitoramento nos sistemas e serviços. Além disso, também não há um acompanhamento especifico voltado para o monitoramento do ambiente em que as aplicações e serviços estão hospedados, ou seja, percebe-se que o gerenciamento de importantes funcionalidades são falhos e que precisam ser melhorados.

Dessa forma, com esse trabalho, espera-se prover meios para realizar a integração das ferramentas utilizadas pelo CPD para o monitoramento, utilizado o protocolo SNMP para facilitar e tornar o gerenciamento dos serviços mais abrangente contribuindo com um bom funcionamento e acompanhamento dos \textit{softwares} da UnB. Além disso pretende-se também que a implementação do protocolo SNMP para monitoramento de serviços possa trazer grandes benefícios como o gerenciamento de falhas, requisições, desempenho e quantidade de acessos em um determinado momento. A partir da implementação, criar e especificar processos, métodos, assim como, realizar estudos e utilizar métricas para estatísticas após a coleta da informações advindas dos serviços implementados para esse propósito. 

%%%%%%%%%%%%%%%%%%%%%%%%%%%%%%%%%%%%%%%

\section{Objetivo Geral}

O objetivo geral do trabalho é propor um \textit{framework} que utilize o protocolo SNMP para o monitoramento de serviços e sistemas distribuídos da UnB. Para isso serão realizadas  pesquisas e projetos desenvolvidos e utilizados para monitoramentos de serviços, técnicas e ferramentas de apoio, visando à melhoria do gerenciamento dos serviços.

\subsection{Objetivos Específicos}

Os objetivos específicos são:

\begin{itemize}
	 
\item Definir um processo para a realização da implementação do protocolo SNMP para monitoramento de serviços baseando-se nas  abordagens mais relevantes estudadas na literatura;

\item Utilizar ferramentas de apoio para a implementação do protocolo SNMP para monitoramento de serviços;

\item implementar o protocolo SNMP para monitoramento de serviços seguindo o RFC\cite{Schoffstall}, documento que especifica os padrões que serão implementados e utilizados em toda a internet ;

\item Realizar um estudo de caso aplicando o método desenvolvido para promover adequações e melhorias no gerenciamento de monitoramento dos serviços da UnB; 

\item Testar, homologar e Implantar a solução proposta para a utilização de monitoramento dos serviços da UnB. 

\end{itemize}

%%%%%%%%%%%%%%%%%%%%%%%%%%%%%%%%%%%%%%%

\section{Resultados Esperados}

O resultado esperado com a execução deste trabalho é a implementação de um \textit{framework} que utilize o protocolo SNMP para o monitoramento de serviços e sistemas distribuídos, e que essa solução após uma avaliação técnica possa ser implantada. Espera-se que o \textit{framework} seja capaz de realizar o monitoramento de sistemas distribuídos em uma arquitetura SOA que sustente a modernização dos sistemas legados da UnB e auxilie no monitoramento e gerenciamento dos novos serviços implementados de forma a garantir o funcionamento dos sistemas e serviços da UnB.

%%%%%%%%%%%%%%%%%%%%%%%%%%%%%%%%%%%%%%%

\section{Metodologia}

O presente trabalho utiliza-se de uma  revisão sistemática para tratar o tema o monitoramento de serviços distribuídos. Este trabalho analisa a utilização de ferramentas de monitoramento de serviços distribuídos com estrutura de base para organizar e formar uma base de conhecimento do tema em questão. Por ser uma das referencias mais citadas em trabalhos sobre RSL, foi utilizado o como referência  para elaboração do trabalho o guia de revisão sistemática da literatura criado por Kitchenham e Charles\cite{kitchenham2007guidelines}.

%%%%%%%%%%%%%%%%%%%%%%%%%%%%%%%%%%%%%%%

\subsection{Questões de pesquisa}

Com intensões e expectativas de se alcançar o objetivo da pesquisa, foram levantadas as seguintes questões de pesquisa \textbf{(QP)}, uma questão principal e outra questão secundária com a intenção de fornecer auxílio na fundamentação da pesquisa, são elas: 

\begin{description}
		\item[QP1)] Análise ou estudo de mecanismos de monitoramento aplicados a sistemas distribuídos?
\item[QP2)] Quais as categorias de monitoramento de sistemas distribuídos que são utilizados nos trabalhos acadêmicos?
\end{description}

As perguntas ou questões têm a intensão de dar subsídio para que durante a busca de informações sobre o tema, seja possível encontrar trabalhos e experiências técnicas que já foram executado e analisados\cite{feltrim2004abordagem}.

%%%%%%%%%%%%%%%%%%%%%%%%%%%%%%%%%%%%%%%

\section{Estrutura do trabalho}

Este trabalho está organizado em quatro capítulos além deste. No Capítulo \ref{mapeamento_sistematico} é apresentada a revisão sistematizada, utilizada na realização de pesquisa para identificação dos principais trabalhos sobre o tema. No Capítulo \ref{proposta} são apresentadas as informações da proposta para implementação do monitoramento de sistemas distribuídos. O Capítulo \ref{consideracoes_finais} são apresentadas as considerações finais, quais os indícios e expectativas geradas no andamento da realização do trabalho.