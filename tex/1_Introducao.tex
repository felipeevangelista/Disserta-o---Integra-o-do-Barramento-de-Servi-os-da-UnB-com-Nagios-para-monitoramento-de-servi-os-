\label{Introducao}

A implementação de sistemas seja ela pela criação, migração ou modernização tecnológica é um assunto que é bastante discutido na \acrfull{UnB}. Nesses últimos anos o assunto com maior repercussão nos departamentos e comunidade acadêmica é a defasagem tecnológica dos \textit{softwares} utilizado para gestão acadêmica e administrativa da \acrshort{UnB}. Na \acrshort{UnB} os sistemas em execução são classificado como \textit{softwares} legados, por não possuir uma documentação, ser de difícil manutenção e também possuir defasagem tecnológica, mas são essenciais para organização. Quando se trata de um processo de modernização, espera-se reduzir os custos com a manutenção dos sistemas legados, criando novos \textit{web services} com a intenção de ampliar a integração entre os sistemas\cite{Agilar}.

O \acrfull{CPD} da \acrshort{UnB}, órgão responsável por desenvolver, gerenciar e manter os sistemas novos e legados, atualmente vem trabalhando na implantação de uma Arquitetura Orientada a Serviços (\acrshort{SOA}), seguindo o modelo de modernização dos \textit{softwares} da \acrshort{UnB} e utilizando como processo o \textit{\acrfull{SMSOC}} para a implementação dos serviços conectando-os  através de um barramento de serviços (\acrshort{ESB}) que utiliza a arquitetura \textit{\acrfull{REST}}.

No entanto, a flexibilidade que a arquitetura proporciona na modernização dos sistemas legados da \acrshort{UnB}, pode dificultar o gerenciamento e monitoramento dos serviços que serão implementados, devido ao modelo negocial de cada sistema, em alguns casos poderão haver situações em que vários serviços e micros serviços serão necessários e determinantes para um único fluxo de negócio do sistema. Diante desse cenário o presente trabalho de pesquisa busca apresentar uma solução para o monitoramento dos serviço implementados, com a implantação do protocolo de monitoramento \textit{\acrfull{SNMP}} como um serviço de monitoramento com o objetivo de facilitar o monitoramento dos serviços com a utilização de um serviço que possibilite a realização da integração com ferramentas de monitoramento, que utilizam o protocolo \acrshort{SNMP} para coletar informações e transmiti-las, assim como indicar condições ou situações de falhas nos serviços em execução monitorados.

%%%%%%%%%%%%%%%%%%%%%%%%%%%%%%%%%%%%%%%%%%%%%%%%%%%%%%%%%%%%%%%%%%%%%%%%%%%%%%%%%%%%%%

\section{Justificativa}

Com o aumento da implementação e disponibilização de serviços na \acrshort{UnB}, foi identificada a necessidade de um efetivo monitoramento dos serviços, através da coleta de dados ou informações extraídas das requisições. Para gerenciar o monitoramento são necessárias ferramentas para um controle mais fácil e objetivo, atualmente o \acrshort{CPD} utiliza ferramentas de mercado com esse propósito, essas ferramentas ou  plataformas são utilizadas para acompanhamento e monitoramento da infraestrutura de redes da \acrshort{UnB}, já a parte de sistemas e serviços não são monitoradas de forma precisa ou que possa trazer informações com certa relevância, pois não há uma comunicação ou integração dos sistemas e serviço de forma apropriada, o que implica em um deficit no acompanhamento e monitoramento nos sistemas e serviços, além disso, também não há um acompanhamento especifico voltado para o monitoramento do ambiente em que as aplicações e serviços estão hospedados, ou seja, percebe-se que o gerenciamento de importantes funcionalidades são falhos e que precisam ser melhorados.

Dessa forma, com esse trabalho, espera-se prover meios para realizar a integração das ferramentas utilizadas pelo \acrshort{CPD} para o monitoramento, utilizando o protocolo \acrshort{SNMP} para integrar e tornar o gerenciamento dos serviços mais abrangente contribuindo com um bom funcionamento e acompanhamento dos \textit{softwares} da \acrshort{UnB}. Pretende-se também que a implantação do protocolo \acrshort{SNMP} para monitoramento de serviços possa trazer grandes benefícios como fornecer métricas sobre gerenciamento de falhas, requisições, desempenho e quantidade de acessos em um determinado momento. A partir da implantação ou implementação quando necessário, criar e especificar processos, métodos, assim como, realizar estudos e utilizar métricas para estatísticas após a coleta das informações advindas dos serviços implementados e disponibilizados para esse propósito. 

%%%%%%%%%%%%%%%%%%%%%%%%%%%%%%%%%%%%%%%%%%%%%%%%%%%%%%%%%%%%%%%%%%%%%%%%%%%%%%%%%%%%%%

\section{Objetivo Geral}

O objetivo geral do trabalho é propor o monitoramento dos sistemas e serviços da \acrshort{UnB} por meio da integração com ferramentas de monitoramento gerenciadas e mantidas pelo \acrshort{CPD} utilizando o protocolo \acrshort{SNMP} para a comunicação. 

\subsection{Objetivos Específicos}
\label{objetivos_especificos}

Os objetivos específicos são:

\begin{itemize}
	 
\item Realizar um mapeamento sistemático para identificar o monitoramento
de sistemas distribuídos;

\item Avaliar as ferramentas de monitoramento existentes e mantidas pelo \acrshort{CPD};

\item Avaliar ferramentas de apoio para a implantação do protocolo \acrshort{SNMP} para integração do monitoramento de serviços;

\item implantar o protocolo \acrshort{SNMP} para monitoramento de serviços seguindo o RFC - 1157\cite{Schoffstall}, documento que especifica os padrões que serão implementados e utilizados em toda a internet;

\item Realizar um estudo de caso aplicando o método desenvolvido para promover adequações e melhorias no gerenciamento de monitoramento dos serviços da \acrshort{UnB}; 

\item Avaliar e analisar os resultados da solução que foi proposta para a realização da integração das ferramentas de monitoramento dos serviços da \acrshort{UnB}. 

\end{itemize}

%%%%%%%%%%%%%%%%%%%%%%%%%%%%%%%%%%%%%%%%%%%%%%%%%%%%%%%%%%%%%%%%%%%%%%%%%%%%%%%%%%%%%%

\section{Resultados Esperados}

O resultado esperado com a execução deste trabalho é a implantação da integração da ferramenta de monitoramento utlizada pelo \acrshort{CPD} com o provedor de serviços denominado Erlangms\cite{Agilar}, utilizada como meio de comunicação para a realização da integração por meio do protocolo \acrshort{SNMP} que comtempla o monitoramento de serviços e sistemas distribuídos, e que essa solução após uma avaliação técnica possa ser implantada. Espera-se que o trabalho proposto seja capaz de realizar o monitoramento de sistemas distribuídos em uma arquitetura \acrshort{SOA} que sustente a modernização dos sistemas legados da \acrshort{UnB} e auxilie no monitoramento e gerenciamento dos novos serviços implementados de forma a garantir o funcionamento da integração entre os sistemas e serviços da \acrshort{UnB}.

%%%%%%%%%%%%%%%%%%%%%%%%%%%%%%%%%%%%%%%%%%%%%%%%%%%%%%%%%%%%%%%%%%%%%%%%%%%%%%%%%%%%%%

\section{Estrutura do trabalho}

Este trabalho está organizado em 4 capítulos, além deste. No Capítulo \ref{fundamentacao_teorica} é apresentada uma fundamentação teórica que abrange temas de contexto importante para o entendimento e execução do trabalho. No Capítulo \ref{mapeamento_sistematico} é apresentado um mapeamento sistemático, utilizado na realização de pesquisa para identificação dos principais trabalhos sobre o tema. No Capítulo \ref{monitoramento_servicos} são apresentadas as informações do trabalho para implantação do monitoramento de serviços por meio do protocolo \acrshort{SNMP}. No Capítulo \ref{conclusao} são apresentadas as conclusões, e quais os indícios e expectativas geradas pelo do trabalho, as contribuições e a possibilidade da realização de trabalhos futuros.