\label{Introducao}

A implementação de sistemas seja ela pela criação, migração ou modernização tecnológica é um assunto que é bastante discutido na \acrfull{UnB}. Nesses últimos anos o assunto com maior repercussão nos departamentos e comunidade acadêmica é a defasagem tecnológica dos \textit{softwares} utilizado para gestão acadêmica e administrativa da \acrshort{UnB}. Na \acrshort{UnB} os sistemas em execução são classificado como \textit{softwares} legados, por não possuir uma documentação, ser de difícil manutenção e também possuir defasagem tecnológica, mas são essenciais para organização. Quando se trata de um processo de modernização, espera-se reduzir os custos com a manutenção dos sistemas legados, criando novos \textit{web services} com a intenção de ampliar a integração entre os sistemas\cite{Agilar}.

O \acrfull{CPD} da \acrshort{UnB}, órgão responsável por desenvolver, gerenciar e manter os sistemas novos e legados, atualmente vem trabalhando na implantação de uma Arquitetura Orientada a Serviços (\acrshort{SOA}), seguindo o modelo de modernização dos \textit{softwares} da \acrshort{UnB} e utilizando como processo o \textit{\acrfull{SMSOC}} para a implementação dos serviços conectando-os  através de um barramento de serviços (\acrshort{ESB}) que utiliza a arquitetura \textit{\acrfull{REST}}.

No entanto, a flexibilidade que a arquitetura proporciona na modernização dos sistemas legados da \acrshort{UnB}, pode dificultar o gerenciamento e monitoramento dos serviços que serão implementados, devido ao modelo negocial de cada sistema, em alguns casos poderão haver situações em que vários serviços e micros serviços serão necessários e determinantes para um único fluxo de negócio do sistema. Diante desse cenário o presente trabalho de pesquisa busca apresentar uma solução para o monitoramento dos serviço implementados, com a implantação do protocolo de monitoramento \textit{\acrfull{SNMP}} como um serviço de monitoramento com o objetivo de facilitar o monitoramento dos serviços com a utilização de um serviço que possibilite a realização da integração com ferramentas de monitoramento, que utilizam o protocolo \acrshort{SNMP} para coletar informações e transmiti-las, assim como indicar condições ou situações de falhas nos serviços em execução monitorados.

%%%%%%%%%%%%%%%%%%%%%%%%%%%%%%%%%%%%%%%%%%%%%%%%%%%%%%%%%%%%%%%%%%%%%%%%%%%

\section{Definição do Problema}

%Com o aumento da implementação e disponibilização de serviços na \acrfull{UnB} por meio do barramento de serviços Erlangms\cite{Agilar}, foi identificada a necessidade de um efetivo monitoramento dos serviços, por meio da coleta de dados ou informações extraídas das requisições durante a execução dos sistemas e serviços. Para gerenciar o monitoramento são necessárias ferramentas que auxiliem essa atividade, o \acrfull{CPD} utiliza ferramentas de mercado com esse propósito, essas ferramentas ou plataformas são utilizadas para acompanhamento e monitoramento da infraestrutura de redes da \acrshort{UnB}.
Com o aumento da implementação e disponibilização de serviços na \acrfull{UnB} por meio do barramento de serviços Erlangms\cite{Agilar}, foi identificada a necessidade de um efetivo monitoramento dos serviços, por meio da coleta de dados ou informações extraídas das requisições durante a execução dos sistemas e serviços. Para gerenciar o monitoramento são necessárias ferramentas que auxiliem essa atividade, o \acrfull{CPD} utiliza ferramentas de mercado com esse propósito, essas ferramentas ou plataformas são utilizadas para acompanhamento e monitoramento da infraestrutura de redes da \acrshort{UnB}.

Os sistemas e serviços mantidos pelo \acrshort{CPD} não são monitoradas de forma precisa e não fornecem informações relevantes sobre a situação, execução, funcionamento e disponibilidade dos sistemas e serviços, pois não há comunicação que possibilite a integração dos sistemas e serviço de forma apropriada com a as ferramentas de monitoramento mantidas pelo \acrshort{CPD}, o que implica em um deficit no acompanhamento e monitoramento nos sistemas e serviços, também não dispõem de acompanhamento especifico voltado para o monitoramento do ambiente em que as aplicações e serviços estão hospedados. Percebe-se que o gerenciamento de importantes funcionalidades são falhos ou ausentes e precisam ser melhorados.

%Dessa forma, com esse trabalho, espera-se prover meios para realizar a integração das ferramentas utilizadas pelo \acrshort{CPD} para o monitoramento, utilizando o protocolo \acrshort{SNMP} para integrar e tornar o gerenciamento dos serviços mais abrangente contribuindo com um bom funcionamento e acompanhamento dos \textit{softwares} da \acrshort{UnB}. Pretende-se também que a implantação do protocolo \acrshort{SNMP} para monitoramento de serviços possa trazer grandes benefícios como fornecer métricas sobre gerenciamento de falhas, requisições, desempenho e quantidade de acessos em um determinado momento. A partir da implantação ou implementação quando necessário, criar e especificar processos, métodos, assim como, realizar estudos e utilizar métricas para estatísticas após a coleta das informações advindas dos serviços implementados e disponibilizados para esse propósito. 

%%%%%%%%%%%%%%%%%%%%%%%%%%%%%%%%%%%%%%%%%%%%%%%%%%%%%%%%%%%%%%%%%%%%%%%%%%%%%%%%%%%%%%

\section{Motivação}



com esse trabalho, espera-se prover meios para realizar a integração das ferramentas utilizadas pelo \acrshort{CPD} para o monitoramento, utilizando o protocolo \acrshort{SNMP} para integrar e tornar o gerenciamento dos serviços mais abrangente contribuindo com um bom funcionamento e acompanhamento dos \textit{softwares} da \acrshort{UnB}. Pretende-se também que a implantação do protocolo \acrshort{SNMP} para monitoramento de serviços possa trazer grandes benefícios como fornecer métricas sobre gerenciamento de falhas, requisições, desempenho e quantidade de acessos em um determinado momento. A partir da implantação ou implementação quando necessário, criar e especificar processos, métodos, assim como, realizar estudos e utilizar métricas para estatísticas após a coleta das informações advindas dos serviços implementados e disponibilizados para esse propósito.


%%%%%%%%%%%%%%%%%%%%%%%%%%%%%%%%%%%%%%%%%%%%%%%%%%%%%%%%%%%%%%%%%%%%%%%%%%%%%%%%%%%%%%

\section{Objetivos}
\label{objetivos}

O objetivo deste trabalho é propor um modelo para o monitoramento dos sistemas e serviços da \acrfull{UnB} com uma solução tecnológica que permita realizar a comunicação por meio de integração com ferramentas de monitoramento utilizadas, gerenciadas e mantidas pelo \acrfull{CPD}.

Para alcançar este objetivo, foram determinados os seguintes objetivos específicos:

\begin{itemize}
	 
\item Realizar um mapeamento sistemático para identificar o monitoramento
de sistemas distribuídos;

\item Avaliar as ferramentas de monitoramento existentes e mantidas pelo \acrshort{CPD};

\item Propor uma arquitetura de integração para monitoramento dos sistemas e serviços da \acrshort{UnB}

\item Implantar o monitoramento dos sistemas e serviços da \acrshort{UnB}

\item Avaliar e analisar os resultados da solução que foi proposta para a realização da integração das ferramentas de monitoramento dos serviços da \acrshort{UnB}. 

\end{itemize}

%%%%%%%%%%%%%%%%%%%%%%%%%%%%%%%%%%%%%%%%%%%%%%%%%%%%%%%%%%%%%%%%%%%%%%%%%%
\section{Metodologia}

O trabalho utiliza-se do mapeamento sistemático para tratar o tema o monitoramento de serviços e sistemas distribuídos. Este trabalho analisa a comunicação por meio da integração com de ferramentas de monitoramento de serviços e sistemas distribuídos para realizar o monitoramento para organizar e formar base de conhecimento sobre monitoramento. Por ser uma das referencias mais citadas em trabalhos sobre mapeamento sistemático  como referência metodológica é utilizado para elaboração do trabalho o guia criado pelos autores Kitchenham e Charles[3].

%%%%%%%%%%%%%%%%%%%%%%%%%%%%%%%%%%%%%%%%%%%%%%%%%%%%%%%%%%%%%%%%%%%%%%%%%%%

\section{Resultados Esperados}

Implantar a integração entre as ferramentas de monitoramento utilizada pelo \acrshort{CPD} e o barramento de serviços denominado Erlangms\cite{Agilar}. Espera-se também que o trabalho proposto possa auxiliar na realização do monitoramento de sistemas e serviços e possibilite monitorar a execução e disponibilidade de sistemas e serviços da \acrshort{UnB}.

%%%%%%%%%%%%%%%%%%%%%%%%%%%%%%%%%%%%%%%%%%%%%%%%%%%%%%%%%%%%%%%%%%%%%%%%%%%%%%%%%%%%%%

\section{Estrutura do Trabalho}

Este trabalho tem a seguinte estrutura. No Capítulo \ref{fundamentacao_teorica} é descrita a fundamentação teórica. No Capítulo \ref{mapeamento_sistematico} é apresentado o mapeamento sistemático, utilizado na realização de pesquisa para identificação dos principais trabalhos sobre o tema. No Capítulo \ref{monitoramento_servicos} são apresentadas as informações do trabalho para implementação e implantação do monitoramento de serviços por meio do protocolo \acrshort{SNMP}. No Capítulo \ref{analise_monitoramento_servicos} são apresentas as informações da avaliação e resultados da implementação do monitoramento do barramento de serviços. No Capítulo \ref{conclusao} são apresentadas as conclusões, e quais os indícios e expectativas geradas pelo do trabalho, as contribuições e a possibilidade da realização de trabalhos futuros.