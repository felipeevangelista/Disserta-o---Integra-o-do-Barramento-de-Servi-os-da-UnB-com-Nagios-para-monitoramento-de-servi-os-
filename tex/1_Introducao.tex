\label{Introducao}

A implementação de sistemas, seja ela pela criação, migração ou modernização tecnológica é um plano utilizado pelas organizações para gerenciar os \textit{softwares} que automatizam seus processos \cite{Agilar}. Quando se trata de um processo de modernização, espera-se reduzir os custos financeiros de mão de obra, com a manutenção dos sistemas em execução que possuem complexidade alta, defasagem tecnológica e a falta de documentação de apoio, com a criação de novos sistemas e serviços. Algumas organizações procuram manter ativos os \textit{softwares} com essas características, com o funcionamento normal e a inclusão de módulos e serviços que possibilitam a integração e comunicação entre serviços e sistemas \cite{Agilar}.

O \acrlong{CPD} da \acrlong{UnB}, órgão responsável por desenvolver, gerenciar e manter os sistemas novos e legados, trabalha na implantação de uma arquitetura \textit{\acrlong{SOA}}, e a utiliza para implementação dos serviços que realizam a comunicação por meio de um barramento de serviços. No entanto, a flexibilidade que a arquitetura \acrshort{SOA} proporciona para modernização dos sistemas novos e legados da \acrshort{UnB}, dificulta o gerenciamento e o monitoramento dos serviços que são implementados \cite{Agilar}. A razão é a quantidade de serviços criados e o modelo negocial de cada sistema, sendo que em alguns casos há situações em que vários serviços e microsserviços são necessários e determinantes para um único fluxo de negócio do sistema.

%%%%%%%%%%%%%%%%%%%%%%%%%%%%%%%%%%%%%%%%%%%%%%%%%%%%%%%%%%%%%%%%%%%%%%%%%%%

\section{Definição do Problema}
Com o aumento da implementação e disponibilização de serviços na \acrshort{UnB} por meio do barramento de serviços Erlangms\cite{Agilar}, foi identificada a necessidade de um efetivo monitoramento dos serviços, por meio da coleta de dados ou informações extraídas das requisições durante a execução dos sistemas e serviços. Para gerenciar o monitoramento são necessárias ferramentas que auxiliem essa atividade, o \acrshort{CPD} utiliza ferramentas de mercado com esse propósito, essas ferramentas ou plataformas são utilizadas para acompanhamento e monitoramento da infraestrutura de redes da \acrshort{UnB}.

Os sistemas e serviços mantidos pelo \acrshort{CPD}, não são monitoradas nem fornecem informações relevantes como por exemplo: a situação, a execução, o funcionamento e disponibilidade dos sistemas e serviços, pois não há integração entre sistemas e serviços de forma apropriada com as ferramentas de monitoramento mantidas pelo \acrshort{CPD}, o que implica em um hiato no acompanhamento e monitoramento nos sistemas e serviços, também não dispõem de acompanhamento específico voltado para o monitoramento do ambiente em que as aplicações e serviços estão hospedados. Percebe-se que o gerenciamento de importantes funcionalidades são falhos ou ausentes e precisam ser melhorados.

%%%%%%%%%%%%%%%%%%%%%%%%%%%%%%%%%%%%%%%%%%%%%%%%%%%%%%%%%%%%%%%%%%%%%%%%%%%

\section{Motivação}

A necessidade de modernizar os \textit{softwares} da \acrshort{UnB} é eminente. A implementação de serviços e microsserviços para contemplar os sistemas novos, legados e adquiridos recebem prioridades para criação. A integração dessas aplicações é construída com diversos tipos de serviços, alguns desses serviços funcionam a partir das informações de outros serviços, a indisponibilidade de um serviço pode impactar em outros serviços, ou até mesmo no sistema como um todo, dessa maneira acompanhar seu funcionamento é essencial.      

É necessária, a criação de meios tecnológicos para realizar o monitoramento via ferramentas de monitoramento utilizadas pelo \acrshort{CPD}, com o barramento de serviços por meio de integração que possibilite a comunicação entre as ferramentas, contribuindo com o acompanhamento abrangente dos serviços e \textit{softwares} da \acrshort{UnB}. Pretende-se com a integração o monitorar serviços. A partir da implantação ou implementação, especificar processos e métodos que possibilite a extração das informações coletadas.


%%%%%%%%%%%%%%%%%%%%%%%%%%%%%%%%%%%%%%%%%%%%%%%%%%%%%%%%%%%%%%%%%%%%%%%%%%%%%%%%%%%%%%

\section{Objetivos}
\label{objetivos}

O objetivo deste trabalho é propor um modelo para o monitoramento dos sistemas e serviços da \acrshort{UnB} com uma solução tecnológica que permita realizar a comunicação por meio de integração com ferramentas de monitoramento utilizadas, gerenciadas e mantidas pelo \acrfull{CPD}.

Para alcançar este objetivo, foram determinados os seguintes objetivos específicos:

\begin{itemize}
	 
\item Realizar uma revisão da literatura para identificar trabalhos sobre monitoramento
de sistemas distribuídos;

\item Propor uma arquitetura de integração para monitoramento dos sistemas e serviços da \acrshort{UnB};

\item Implantar o monitoramento dos sistemas e serviços da \acrshort{UnB};

\item Avaliar e analisar os resultados da solução que foi proposta para a realização da integração das ferramentas de monitoramento dos serviços da \acrshort{UnB}. 

\end{itemize}

%%%%%%%%%%%%%%%%%%%%%%%%%%%%%%%%%%%%%%%%%%%%%%%%%%%%%%%%%%%%%%%%%%%%%%%%%%
\section{Metodologia}

A metodologia de pesquisa utilizada neste trabalho será a revisão da literatura em conjunto
com um estudo de caso exploratório. A revisão da literatura objetiva reconhecer a unidade e a diversidade interpretativa existente no eixo temático em que se insere o problema
em estudo, para ampliar, ramificar a análise interpretativa, bem como para compor
as abstrações e sínteses que qualquer pesquisa requer colaborando para a coerência nas
argumentações do pesquisador \cite{petticrew2008systematic}. Ao final da revisão da literatura, espera-se obter um conjunto de indicadores que serão utilizados na aplicação do modelo proposto.

O estudo de caso exploratório tem como finalidade proporcionar mais informações
sobre o assunto a ser investigado, possibilitando sua definição e seu delineamento, isto é,
facilitar a delimitação do tema da pesquisa; orientar a fixação dos objetivos e a formulação
das hipóteses ou descobrir um novo tipo de enfoque para o assunto \cite{de2018mapeamento}.

%%%%%%%%%%%%%%%%%%%%%%%%%%%%%%%%%%%%%%%%%%%%%%%%%%%%%%%%%%%%%%%%%%%%%%%%%%%

\section{Resultados Esperados}

Implantar a integração entre as ferramentas de monitoramento utilizada pelo \acrshort{CPD} e o barramento de serviços denominado Erlangms\cite{Agilar}. Espera-se também que este trabalho possa auxiliar na realização do monitoramento de sistemas e serviços, e possibilite monitorar a execução e disponibilidade de sistemas e serviços da \acrshort{UnB}.

%%%%%%%%%%%%%%%%%%%%%%%%%%%%%%%%%%%%%%%%%%%%%%%%%%%%%%%%%%%%%%%%%%%%%%%%%%%%%%%%%%%%%%

\section{Estrutura do Trabalho}

O presente trabalho de pesquisa apresenta uma solução para o monitoramento dos serviços implementados, com a implementação e implantação de um módulo de monitoramento de serviços. Com essa solução espera-se proporcionar o monitoramento dos serviços por meio de ferramentas de monitoramento com a utilização de um protocolo de comunicação capaz de permitir a integração entre o barramento de serviços e ferramentas de monitoramento, para coletar informações e transmiti-las, esses dados sobre monitoramento podem indicar o funcionamento normal, condições de alertas ou situações de falhas nos serviços em execução monitorados.

Este trabalho está assim estruturado: no capítulo \ref{fundamentacao_teorica} é descrita a fundamentação teórica, no capítulo \ref{mapeamento_sistematico} é apresentado o mapeamento sistemático, utilizado na realização de pesquisa para identificação dos principais trabalhos sobre o tema, no capítulo \ref{monitoramento_servicos} são apresentadas a implementação e a implantação do monitoramento de serviços, no capítulo \ref{analise_monitoramento_servicos} são apresentados a avaliação e os resultados da implementação do monitoramento do barramento de serviços e, finalmente no capítulo \ref{conclusao} são apresentadas as conclusões, as contribuições e a possibilidade da realização de trabalhos futuros.