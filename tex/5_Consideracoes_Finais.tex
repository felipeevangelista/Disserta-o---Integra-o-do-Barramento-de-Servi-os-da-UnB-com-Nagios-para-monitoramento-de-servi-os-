\label{consideracoes_finais}

O \acrshort{CPD} vem trabalhando nos últimos anos na modernização dos \textit{softwares} por ele gerenciados, desde os sistemas que eram hospedados nos \textit{mainframes}, que precisaram ser migrados para outras plataformas por causa da virada do milênio (\textit{bug} do milênio) até os dias atuais. 

Nessa década optou-se algumas vezes pela implementação de \textit{frameworks}, com abstração integral no controle de acesso das interfaces de usuário, iniciando na camada de persistência, passando pela camada de negócio até a camada de visão, ou seja \textit{frameworks} com alto acoplamento e baixa coesão. 

No entanto, com a experiencia adquirida ao longo dos anos, com erros e acertos em definições arquiteturais, finalmente por meio de pesquisas e fundamentação teórica, decidiu-se pela utilização de uma arquitetura \acrshort{SOA} que é leve, flexível, coesa e com baixo acoplamento, possui seus desafios e \textit{tradeoffs}, mas que devido a sua padronização é muito bem aceita no mercado, instituições e empresas de desenvolvimento de \textit{software} e proporcionando um desenvolvimento mais ágil dos serviços, por parte dos desenvolvedores.    

Nesse sentido, o trabalho teve com objetivo propor implementação de um \textit{framework} de monitoramento com abordagens citada no capítulo \ref{proposta}, que forneça apoio no processo de modernização dos \textit{softwares} da UnB.
Com a implementação do \textit{framework} de monitoramento espera-se que os \textit{web services} da plataforma possam ser monitorados de modo tornar o funcionamento dos serviços o mais adequado possível.  

Uma consideração que deve ser levada em conta nesse trabalho é a facilidade da realização de validação da proposta, visto que atualmente com o processo de modernização dos sistemas legado com a utilização arquitetura \acrshort{SOA}, os serviços (\textit{web services}) devem ser acompanhados e monitorados, pois já começam a tornar-se fundamentais, devido a utilização em outras plataformas e sistemas, por causa da flexibilidade e fácil integração. 

O trabalho tem a intenção de suprir uma necessidade em relação a gestão, acompanhamento e monitoramento dos serviços que englobam um conjunto de \textit{web services} que compõem os sistemas, e que vai ao encontro dos projetos de desenvolvimento de \textit{software} do \acrshort{CPD} a fim tornar o processo de modernização dos \textit{softwares} da \acrshort{UnB} uma realidade.
