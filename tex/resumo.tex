A implementação de serviços e microsserviços para aplicações de sistemas distribuídos com a utilização de uma Arquitetura Orientada a Serviços (\acrshort{SOA}) permite utilizar padrões de desenvolvimento, facilitar a manutenção, flexibilizar o desenvolvimento de serviços e permitir a interoperabilidade de serviços e sistemas. O \acrfull{CPD} da \acrfull{UnB} trabalha com vários processos de automação de \textit{softwares}, desde a manutenção de sistemas legados, passando pelo desenvolvimento de novas aplicações até a implantação de \textit{softwares} adquiridos, com várias frentes tecnológicas relacionadas à sistemas. Acompanhar e monitorar o funcionamento de serviços, microsserviços e sistemas é imprescindível. Este trabalho tem caráter exploratório e busca investigar sobre soluções e ferramentas para implementação e implantação de monitoramento de serviços e sistemas distribuídos da \acrfull{UnB}, por meio de uma revisão da literatura. Com embasamento teórico obteve-se um modelo que foi implementado como módulo de monitoramento do barramento de serviços da \acrfull{UnB}. Neste trabalho foram executadas simulações na solução que permitiu analisar a integração do barramento de serviços com a ferramenta de monitoramento através da solução proposta.