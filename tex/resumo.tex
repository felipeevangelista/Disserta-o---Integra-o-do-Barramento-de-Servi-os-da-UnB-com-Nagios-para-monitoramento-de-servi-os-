A implementação de serviços e micro serviços para aplicações de sistemas distribuídos, vem ganhando espaço na escolha de arquitetos de \textit{software} para a aplicação e definição de uma Arquitetura Orientada a Serviços (SOA), devido a fácil implementação que fornece modularidade e flexibilidade no desenvolvimento. No Centro de Informática (CPD) da Universidade de Brasília (UnB) surge como proposta para a modernização dos sistemas existentes, mantidos e gerenciados pelo CPD, esses sistemas possuem grande visibilidade estratégica e são peças chave para a aplicação e automação dos processos da UnB, por esses motivos são bastante utilizados. Os sistemas apresentam algumas características de sistemas legados como a defasagem tecnológica e falta de documentação, por esses motivos, vê-se a importância da modernização desses sistemas. A proposta adotada pelo CPD é o desenvolvimento de \textit{web services} para modernização dos sistemas de forma modular, levantando, definindo e priorizando os serviços ou micro serviços, de maneira a disponibilizar \textit{web services} capazes de substituir algumas funcionalidades dos sistemas existentes, os sistemas possuem várias funcionalidades, e por isso, muitos \textit{web services} foram desenvolvidos e devido a flexibilidade arquitetural provavelmente muitos outros \textit{web services} serão desenvolvidos. No entanto surge como problema, devido o grande número de \textit{web services} desenvolvidos, o gerenciamento e monitoramento desses \textit{web services}. O trabalho tem o objetivo de investigar soluções e ferramentas para a implementação do Monitoramento de sistemas distribuídos utilizando o Protocolo \textit{Simple Network Management Protocol (SNMP)}, com a intenção de monitorar o funcionamento dos \textit{web services}, de modo a permitir o acompanhamento, como o número de requisições e sua disponibilidade. 