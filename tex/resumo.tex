A implementação de serviços e micro serviços para aplicações de sistemas distribuídos, vem ganhando espaço na escolha de arquitetos de \textit{software} para a aplicação e definição de uma Arquitetura Orientada a Serviços (\acrshort{SOA}), devido a possibilidade de organização e uso de padrões que permitem e fornecem aos desenvolvedores a modularidade, facilidade de manutenção, flexibilidade no desenvolvimento de serviços(\textit{web services}), reutilização e interoperabilidade de serviços desenvolvidos. O \acrfull{CPD} da \acrfull{UnB} atualmente vem trabalho em vários processos de automação de \textit{softwares}, que vem desde a manutenção de sistemas legados, passando pelo desenvolvimento de novas aplicações até a implantação de \textit{softwares} adquiridos, ou seja trabalhando com várias frentes tecnológicas relacionadas a sistemas. A proposta adotada pelo \acrshort{CPD} é a implantação de \textit{softwares} adquiridos e o desenvolvimento de \textit{web services} para modernização e integração dos sistemas, de maneira a disponibilizar \textit{web services} capazes de substituir algumas funcionalidades dos sistemas existentes(sistemas legados), assim com garantir a interoperabilidade entre eles, pois os sistemas possuem várias funcionalidades, e por isso, muitos \textit{web services} necessariamente foram desenvolvidos, e que devido a flexibilidade arquitetural e a meta do \acrshort{CPD} diante da necessidade de implantação, implementação e integração, provavelmente muitos outros \textit{web services} serão desenvolvidos. No entanto surge como problema, devido o grande número de \textit{web services} desenvolvidos, o gerenciamento e monitoramento desses \textit{web services}. O trabalho tem o objetivo de investigar soluções e ferramentas para a implementação do Monitoramento de sistemas distribuídos utilizando o Protocolo \textit{\acrfull{SNMP}}, com a intenção de monitorar o funcionamento dos \textit{web services}, de modo a permitir o acompanhamento e o gerenciamento de métricas, como por exemplo o número de requisições de um determinado \textit{web service} e a sua disponibilidade. 