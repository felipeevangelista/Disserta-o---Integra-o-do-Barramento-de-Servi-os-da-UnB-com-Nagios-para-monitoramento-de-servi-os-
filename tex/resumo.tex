A implementação de serviços e micro serviços para aplicações de sistemas distribuídos com a utilização de uma Arquitetura Orientada a Serviços (\acrshort{SOA}) fornece a possibilidade para utilizar padrões de desenvolvimento, facilitar a manutenção, flexibilizar o desenvolvimento de serviços e permitir a interoperabilidade de serviços e sistemas. O \acrfull{CPD} da \acrfull{UnB} trabalha com vários processos de automação de \textit{softwares}, desde a manutenção de sistemas legados, passando pelo desenvolvimento de novas aplicações até a implantação de \textit{softwares} adquiridos, com várias frentes tecnológicas relacionadas a sistemas. Acompanhar e monitorar o funcionamento de serviços, micros serviços e sistemas é imprescindível. Este trabalho tem caráter exploratório e busca investigar sobre soluções e ferramentas para a implementação e implantação de monitoramento de serviços e sistemas distribuídos da \acrfull{UnB} embasado em um mapeamento sistemático de monitoramento de sistemas distribuídos a fim de apoiar no monitoramento durante o funcionamento desses serviços e aplicações. A partir da realização do mapeamento sistemático obteve-se uma solução que possibilitou criar um módulo no barramento de serviços da \acrfull{UnB} para realizar o monitoramento. Neste trabalho foram realizados testes e simulações na solução que permitiu verificar e validar a integração entre o barramento de serviços e uma ferramenta de monitoramento por meio de um protocolo.