\label{analise_monitoramento_servicos}

Com a realização da implementação, implantação e integração das aplicações para a execução do monitoramento, percebeu-se a necessidade de divulgação da análise e resultados do trabalho executado. Este Capítulo apresenta a análise realizada após a execução do monitoramento do barramento de serviços por meio da ferramenta de monitoramento com a utilização do protocolo \acrshort{SNMP}.

%%%%%%%%%%%%%%%%%%%%%%%%%%%%%%%%%%%%%%%%%%%%%%%%%%%%%%%%%%%%%%%

\section{Análise das Aplicações}
\label{analise}
Com o propósito de avaliar a execução do trabalho foram analisados alguns requisitos das plataformas utilizadas para a realização do monitoramento do barramento de serviços por meio de uma ferramenta de monitoramento. Inicialmente os requisitos dos recursos computacionais das aplicações foram avaliados, tanto o requisito do barramento de serviços quanto o da ferramenta de monitoramento. O agente de monitoramento utiliza os mesmos recuros computacionais do \textit{software} onde ele foi implantado, nesse caso agente de monitoramento é representado pelo \textit{software} Exometer e está implantado no barramento de serviços Erlangms, os requisitos de instalação do barramento de serviços poderão ser visualizados em \cite{erlangms_gitHub}, assim como os requisitos de configuração do Exometer poderão ser vizualizados em \cite{exometer_gitHub} . A ferramenta de monitoramento também dispões de requisitos básicos para o seu funcionamento, os requisitos do Nagios Core\textsuperscript{\textregistered} na versão 4 poderão ser visualizados em \cite{nagios_core_configuration}, em ambos os caso os requisitos computacionais não impediram a execução do projeto e consequentemente foram executados sem gerar grandes esforços.

Dispondo da configuração das aplicações foi necessário realizar a avaliação do funcionamento do monitoramento por meio do conjunto de protocolos TCP/IP, onde sabe-se que o protocolo \acrshort{SNMP} é um protocolo da \textit{Internet}, que por sua vez o protocolo \acrshort{SNMP} para garantir a transmissão dos dados de monitoramento utiliza por padrão o protocolo UDP especificamente nas portas 161 e 162, na porta 161 estão associadas a todas as mensagens enviadas ao protocolo \acrshort{SNMP} e a porta 162 é utilizada para realização das interceptações de todas as mensagens que transmitirem \textit{TRAPs}, lembrando que de acordo com as especificações da RFC 1157, essas mensagens não serão aceitas caso o tamanho exceda 484 octetos\cite{Schoffstall}. 

Dando continuidade na avaliação, percebeu-se que as configurações de agente e gerente \acrshort{SNMP} tem o fator presencial de muita importância, nesse trabalho o agente é quem transmite as informações pela 4000 e o gerente pela porta 5000, durante a execução do trabalho as configurações das portas ocasionaram um pequeno problema, pois durante o funcionamento do monitoramento no ambiente de desenvolvimento, mesmo que as aplicações possuíssem \acrshort{IP}s distintos, as portas geradas com identificações diferentes não funcionariam, pois as portas devem possuir as mesmas identificações tanto para transmitir quanto para interceptar, ou seja, se qualquer transmissão que for realizada pela porta 162 essa transmissão deverá ser interceptada também na porta 162, quando as aplicações são instaladas na mesma máquina a primeira aplicação que inciar será a detentora da porta, impossibilitando assim o funcionamento das aplicações em um único servidor, validando assim que cada \textit{host} necessita ter a sua configuração individualizada e preparada para que o agente seja responsável pela transmissão dos \textit{TRAPs}. 

Uma importante avaliação para a realização do monitoramento do barramento de serviços, foi avaliação feita sobre a construção do arquivo MIB. Na leitura de trabalhos técnicos e verificações em sites de empresas especializadas no ramo de monitoramento o arquivo MIB normalmente é estático, pois contempla especificamente os \acrshort{OID}s para a realização do monitoramento dessas aplicações ou ativos de rede, Nesse trabalho a instrumentação provida pelo barramento de serviços gera informações a partir de sua execução, sendo necessário um dinamismo na criação de objetos, o Exometer com funções bem definidas e de fácil compreensão correspondeu as expectativas geradas a partir de sua documentação, contribuindo bastante na criação e atualização do arquivo MIB, na criação de métricas, atualização de valores e no \textit{report} dos dados coletados.  

A ferramenta de monitoramento utilizada para a execução do trabalho demonstrou pontos em que o dinamismo alcançado pela aplicação do agente de monitoramento não poderia ser utilizada de imediato. Após uma breve análise da ferramenta de monitoramento, percebeu-se que ela funciona a partir de módulos(\textit{plugins}) complementares, dependendo do tipo de monitoramento a ser realizado, nesse caso o monitoramento que foi realizado foi em modo passivo, para isso foram realizadas as configurações dos \textit{plugins} \acrshort{SNMPTT} e o SNMPTRAPD que necessitam dos \acrshort{OID}s mapeados, ou seja, para que a comunicação e transmissão dos dados do monitoramento sejam tratados como confiáveis é necessário informar de qual \textit{host} e quais objetos terão as mensagens interceptadas.Esse foi um ponto que não era esperado no projeto, mas que foi de grande utilidade para entender e auxiliar na definição de como realizar o monitoramento em modo passivo.   

Por fim, mediante a realização da avaliação e utilização das ferramentas podemos confirmar a partir do estudo e experimento realizado, a possibilidade da realização do monitoramento dos recursos do barramento de serviços por meio de uma ferramenta de monitoramento com a utilização do protocolo \acrshort{SNMP}, de acordo com a realização do experimento pode-se comprovar que poderá ser utilizada qualquer ferramenta de monitoramento que permita a integração e interceptação de \textit{TRAPs} \acrshort{SNMP}, a implantação do agente \acrshort{SNMP} permite incluir a funcionalidade com uma das principais características do barramento de serviços Erlangms.   


%%%%%%%%%%%%%%%%%%%%%%%%%%%%%%%%%%%%%%%%%%%%%%%%%%%%%%%%%%%%%%%

\section{Resultados}
\label{resultados}

A partir da realização do experimento que possibilitou a integração das aplicações para a execução do monitoramento dos recursos do barramento, esse trabalho demonstra atender a expectativa gerada pela utilização do protocolo \acrshort{SNMP} para elaborar a comunicação com a ferramenta de monitoramento, inicialmente um dos pontos que foi muito importante para contemplar o monitoramento e se demonstrou com resultado satisfatório, foi a criação dinâmica do arquivo MIB, o arquivo MIB com a estrutura inicial poderá ser visualizado na figura \ref{fun:fig:mib-original} e o arquivo MIB com os \acrshort{OID}s criados durante a execução do barramento de serviços poderá ser visualizado na figura \ref{fun:fig:mib-metricas}.

\begin{figure}[H]
    \begin{minted}
    [frame=lines,framesep=2mm,baselinestretch=1.0,linenos]{python}
    EXOMETER-METRICS-MIB DEFINITIONS ::= BEGIN

IMPORTS
    MODULE-IDENTITY, OBJECT-TYPE, NOTIFICATION-TYPE,
    Counter32, Counter64, Gauge32, Integer32, 
    snmpModules, experimental FROM SNMPv2-SMI
    MODULE-COMPLIANCE, OBJECT-GROUP, NOTIFICATION-GROUP 
    FROM SNMPv2-CONF;

exometerMetricsMIB MODULE-IDENTITY
	LAST-UPDATED "201401190525Z"
	ORGANIZATION "Feuerlabs"
	CONTACT-INFO "TODO" 
	DESCRIPTION 
		"This MIB module is used for exposing dynamic
		exometer metrics."
	REVISION  "201401190525Z"
	DESCRIPTION 
		"The initial version"
	::= { snmpModules 1 }

exometerMetrics OBJECT IDENTIFIER ::= { experimental 7 }

-- CONTENT START

-- CONTENT END

END
    \end{minted}
    \caption{O arquivo MIB sem os OIDs.}
    \label{fun:fig:mib-original}
\end{figure} 

\begin{frame}{}
    

\begin{minted}
    [frame=lines,framesep=2mm,baselinestretch=1.0,linenos]{python}
   EXOMETER-METRICS-MIB DEFINITIONS ::= BEGIN

IMPORTS
    MODULE-IDENTITY, OBJECT-TYPE, NOTIFICATION-TYPE,
    Counter32, Counter64, Gauge32, Integer32, 
    snmpModules, experimental FROM SNMPv2-SMI
    MODULE-COMPLIANCE, OBJECT-GROUP, NOTIFICATION-GROUP 
    FROM SNMPv2-CONF;

exometerMetricsMIB MODULE-IDENTITY
	LAST-UPDATED "201401190525Z"
	ORGANIZATION "Feuerlabs"
	CONTACT-INFO "TODO" 
	DESCRIPTION 
		"This MIB module is used for exposing dynamic
		exometer metrics."
	REVISION  "201401190525Z"
	DESCRIPTION 
		"The initial version"
	::= { snmpModules 1 }

exometerMetrics OBJECT IDENTIFIER ::= { experimental 7 }

-- CONTENT START

-- METRIC datapointEmsLoggerWriteErrorValue START
datapointEmsLoggerWriteErrorValue OBJECT-TYPE
    SYNTAX Counter32
    MAX-ACCESS read-only
    STATUS current
    DESCRIPTION ""
    ::= { exometerMetrics 1 }
-- METRIC datapointEmsLoggerWriteErrorValue END

-- METRIC datapointEmsLoggerWriteErrorMsSinceReset START
datapointEmsLoggerWriteErrorMsSinceReset OBJECT-TYPE
    SYNTAX Counter32
    MAX-ACCESS read-only
    STATUS current
    DESCRIPTION ""
    ::= { exometerMetrics 3 }
-- METRIC datapointEmsLoggerWriteErrorMsSinceReset END

-- METRIC datapointEmsLoggerWriteInfoValue START
datapointEmsLoggerWriteInfoValue OBJECT-TYPE
    SYNTAX Counter32
    MAX-ACCESS read-only
    STATUS current
    DESCRIPTION ""
    ::= { exometerMetrics 4 }
-- METRIC datapointEmsLoggerWriteInfoValue END

-- METRIC datapointEmsLoggerWriteInfoMsSinceReset START
datapointEmsLoggerWriteInfoMsSinceReset OBJECT-TYPE
    SYNTAX Counter32
    MAX-ACCESS read-only
    STATUS current
    DESCRIPTION ""
    ::= { exometerMetrics 5 }
-- METRIC datapointEmsLoggerWriteInfoMsSinceReset END

-- METRIC datapointEmsLoggerWriteWarnValue START
datapointEmsLoggerWriteWarnValue OBJECT-TYPE
    SYNTAX Counter32
    MAX-ACCESS read-only
    STATUS current
    DESCRIPTION ""
    ::= { exometerMetrics 6 }
-- METRIC datapointEmsLoggerWriteWarnValue END

-- METRIC datapointEmsLoggerWriteWarnMsSinceReset START
datapointEmsLoggerWriteWarnMsSinceReset OBJECT-TYPE
    SYNTAX Counter32
    MAX-ACCESS read-only
    STATUS current
    DESCRIPTION ""
    ::= { exometerMetrics 7 }
-- METRIC datapointEmsLoggerWriteWarnMsSinceReset END

-- OBJECT-GROUP allObjects START
allObjects OBJECT-GROUP
    OBJECTS {
        datapointEmsLoggerWriteErrorMsSinceReset,
        datapointEmsLoggerWriteErrorValue,
        datapointEmsLoggerWriteInfoMsSinceReset,
        datapointEmsLoggerWriteInfoValue,
        datapointEmsLoggerWriteWarnMsSinceReset,
        datapointEmsLoggerWriteWarnValue
    }
    STATUS current
    DESCRIPTION ""
    ::= { exometerMetrics 2 }
-- OBJECT-GROUP allObjects END

-- INFORM reportEmsLoggerWriteErrorValue START
reportEmsLoggerWriteErrorValue NOTIFICATION-TYPE
    OBJECTS {
        datapointEmsLoggerWriteErrorValue
    }
    STATUS current
    DESCRIPTION ""
    ::= { exometerMetrics 8 }
-- INFORM reportEmsLoggerWriteErrorValue END

-- INFORM reportEmsLoggerWriteInfoValue START
reportEmsLoggerWriteInfoValue NOTIFICATION-TYPE
    OBJECTS {
        datapointEmsLoggerWriteInfoValue
    }
    STATUS current
    DESCRIPTION ""
    ::= { exometerMetrics 10 }
-- INFORM reportEmsLoggerWriteInfoValue END

-- INFORM reportEmsLoggerWriteWarnValue START
reportEmsLoggerWriteWarnValue NOTIFICATION-TYPE
    OBJECTS {
        datapointEmsLoggerWriteWarnValue
    }
    STATUS current
    DESCRIPTION ""
    ::= { exometerMetrics 11 }
-- INFORM reportEmsLoggerWriteWarnValue END

-- NOTIFICATION-GROUP allNotifications START
allNotifications NOTIFICATION-GROUP
    NOTIFICATIONS {
        reportEmsLoggerWriteErrorValue,
        reportEmsLoggerWriteInfoValue,
        reportEmsLoggerWriteWarnValue
    }
    STATUS current
    DESCRIPTION ""
    ::= { exometerMetrics 9 }
-- NOTIFICATION-GROUP allNotifications END

-- CONTENT END

END
\end{minted}
\begin{center}
\caption{Figura \ref{fun:fig:mib-metricas}: O arquivo MIB com os
OIDs.}    
\end{center}
\label{fun:fig:mib-metricas}
\end{frame}  


 
%%%%%%%%%%%%%%%%%%%%%%%%%%%%%%%%%%%%%%%%%%%%%%%%%%%%%%%%%%%%%%%

\section{Síntese do Capítulo}
\label{sintese5}