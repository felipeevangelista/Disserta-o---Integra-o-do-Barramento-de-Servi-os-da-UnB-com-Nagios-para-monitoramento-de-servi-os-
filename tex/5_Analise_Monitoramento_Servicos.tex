\label{analise_monitoramento_servicos}

Com a realização da implementação, implantação e integração das aplicações para a execução do monitoramento, percebeu-se a necessidade de divulgação da análise e resultados do trabalho executado. Este Capítulo apresenta a análise realizada após a execução do monitoramento do barramento de serviços por meio da ferramenta de monitoramento com a utilização do protocolo \acrshort{SNMP}.

%%%%%%%%%%%%%%%%%%%%%%%%%%%%%%%%%%%%%%%%%%%%%%%%%%%%%%%%%%%%%%%

\section{Análise das Aplicações}
\label{analise}
Com o propósito de avaliar a execução do trabalho foram analisados alguns requisitos das plataformas utilizadas para a realização do monitoramento do barramento de serviços por meio de uma ferramenta de monitoramento. Inicialmente os requisitos dos recursos computacionais das aplicações foram avaliados, tanto o requisito do barramento de serviços quanto o da ferramenta de monitoramento. O agente de monitoramento utiliza os mesmos recuros computacionais do \textit{software} onde ele foi implantado, nesse caso agente de monitoramento é representado pelo \textit{software} Exometer e está implantado no barramento de serviços Erlangms, os requisitos de instalação do barramento de serviços poderão ser visualizados em \cite{erlangms_gitHub}. A ferramenta de monitoramento também dispões de requisitos básicos para o seu funcionamento, os requisitos do Nagios Core\textsuperscript{\textregistered} na versão 4 poderão ser visualizados em \cite{nagios_core_configuration}, em ambos os caso os requisitos computacionais não impediram a execução do projeto e consequentemente foram executados sem gerar grandes esforços.

Dispondo da configuração das aplicações foi necessário realizar a avaliação do funcionamento do monitoramento por meio do conjunto de protocolos TCP/IP, onde sabe-se que o protocolo \acrshort{SNMP} é um protocolo da \textit{Internet}, que por sua vez o protocolo \acrshort{SNMP} para garantir a transmissão dos dados de monitoramento utiliza por padrão o protocolo UDP especificamente nas portas 161 e 162, na porta 161 estão associadas a todas as mensagens enviadas ao protocolo \acrshort{SNMP} e a porta 162 é utilizada para realização das interceptações de todas as mensagens que transmitirem \textit{TRAPs}, lembrando que de acordo com as especificações da RFC 1157, essas mensagens não serão aceitas caso o tamanho exceda 484 octetos\cite{Schoffstall}. 

Dando continuidade a avaliação as configurações de agente e gerente \acrshort{SNMP} tem o fator presencial de muita importância, nesse trabalho o agente é quem transmite as informações pela 4000 e o gerente pela porta 5000, durante a execução do trabalho as configurações das portas ocasionaram um pequeno problema, pois durante o funcionamento do monitoramento no ambiente de desenvolvimento, mesmo que as aplicações possuíssem \acrshort{IP}s distintos, as portas geradas com identificações diferentes não funcionariam, pois as portas devem possuir as mesmas identificações tanto para transmitir quanto para interceptar, ou seja, se qualquer transmissão for realizada pela porta 162 essa transmissão deverá ser interceptada também na porta 162, quando as aplicações são instaladas na mesma máquina a primeira aplicação que inciar será a detentora da porta, impossibilitando assim o funcionamento das aplicações em um único servidor, validando assim que cada \textit{host} necessita ter a sua configuração individualizada e preparada para que o agente seja responsável pela transmissão dos \textit{TRAPs}. 

Uma importante avaliação para a realização do monitoramento do barramento de serviços, foi avaliação feita sobre a construção do arquivo MIB. Na leitura de trabalhos técnicos e verificações em sites de empresas especializadas no ramo de monitoramento o arquivo MIB normalmente é estático, pois contempla especificamente os \acrshort{OID}s para a realização do monitoramento dessas aplicações ou ativos de rede, Nesse trabalho a instrumentação provida pelo barramento de serviços gera informações a partir de sua execução, sendo necessário um dinamismo na criação de objetos, o Exometer com funções bem definidas e de fácil compreensão correspondeu as expectativas geradas a partir de sua documentação, contribuindo bastante na criação de métricas, atualização de valores e no \textit{report} dos dados coletados.  

A ferramenta de monitoramento utilizada para a execução do trabalho demonstrou pontos em que o dinamismo alcançado pela aplicação do agente de monitoramento não poderia ser utilizada de imediato. Após uma breve análise da ferramenta de monitoramento, percebeu-se que ela funciona a partir de módulos(\textit{plugins}) complementares, dependendo do tipo de monitoramento a ser realizado, nesse caso o monitoramento que foi realizado foi em modo passivo, para isso foram realizadas as configurações dos \textit{plugins} \acrshort{SNMPTT} e o SNMPTRAPD que necessitam dos \acrshort{OID}s mapeados, ou seja, para que a comunicação e transmissão dos dados do monitoramento sejam relatados como confiáveis é necessário informar de qual \textit{host} e quais objetos terão as mensagens interceptadas.Esse foi um ponto que não era esperado no projeto, mas que foi de grande utilidade para entender e auxiliar na definição de como realizar o monitoramento em modo passivo.   




%%%%%%%%%%%%%%%%%%%%%%%%%%%%%%%%%%%%%%%%%%%%%%%%%%%%%%%%%%%%%%%

\section{Resultados}
\label{resultados}

%%%%%%%%%%%%%%%%%%%%%%%%%%%%%%%%%%%%%%%%%%%%%%%%%%%%%%%%%%%%%%%

\section{Síntese do Capítulo}
\label{sintese5}